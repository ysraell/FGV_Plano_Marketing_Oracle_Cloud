%%%%%%%%%%%%%%%%%%%%%%%%%%%%%%%%%%%%%%%%%%%%%%%%%%%%%%%%%%%%%%%%%%%%%%%
% How to use writeLaTeX: 
%
% You edit the source code here on the left, and the preview on the
% right shows you the result within a few seconds.
%
% Bookmark this page and share the URL with your co-authors. They can
% edit at the same time!
%
% You can upload figures, bibliographies, custom classes and
% styles using the files menu.
%
%%%%%%%%%%%%%%%%%%%%%%%%%%%%%%%%%%%%%%%%%%%%%%%%%%%%%%%%%%%%%%%%%%%%%%
\documentclass[a4paper]{article}

\usepackage{renote-template}
\usepackage{graphicx,url}
\usepackage[brazil]{babel}
\usepackage[utf8]{inputenc}  
\usepackage{longtable}
\usepackage[none]{hyphenat}
\usepackage{fancyhdr}
\usepackage{setspace}
\usepackage{hyperref}
\usepackage[alf, bibjustif, abnt-etal-list=5, abnt-emphasize=bf,
            abnt-etal-text=it, abnt-repeated-author-omit=no        ]{abntex2cite}

\fancyhf{}
\fancyhead[C]{\thepage}
\pagestyle{fancy}
\sloppy
\singlespacing
\newcommand{\C}{\emph{Cloud}}

\title{Plano de Marketing: ORACLE \\
Agosto, 2020
}

\author{
    Israel Gonçalves de Oliveira (prof.israel@gmail.com) \\
    MBA Executivo: Gestão com Ênfase em Liderança e Inovação \\
    Turma: ONL020GI-ENFLICB01T1 \\
    Professora Solimar Garcia}

\begin{document} 

\maketitle

\section{Introdução}

O presente documento tem como objetivo apresentar um plano estratégico de marketing visando a alavancagem do uso dos serviços de computação em nuvem da empresa de tecnologia Oracle, tendo como foco o, mas não se limitando apenas ao, mercado brasileira. Também é apresentado uma breve descrição da empresa e do serviços de computação em nuvem de  por ela prestados. São elencadas as principais características da empresa e do mercado, dados relevantes sobre a ambientação e uma análise \emph{SWOT}. Com efeito, é apresentado um conjuntos de ações (estratégia) de marketing visando o aumento da popularidade dos serviços da Oracle entre os profissionais da área de tecnologia, ou seja, visando um \emph{buzz} marketing dentro das empresas (para promover o B2B).

\subsection{Breve história da Oracle}
A \emph{Oracle Corporation} é uma multinacional com origem nos EUA, fundada em 1977 com o nome \emph{Software Development Laboratories} por Larry Ellison (1944), conselheiro executivo e atual CTO (\emph{chief technology officer}), Bob Miner (1941-1994), falecido por complicações devido a um câncer de pulmão, e Ed Oates (1946) atualmente atua como empresário e investidor (chegou a ser guitarrista e apresentou-se em shows). Pode-se resumir a história da Oracle olhando para o desenvolvimento de seu principal produto por décadas, seu sistema gerenciador de banco de dados relacional (SGBDR) - do inglês \emph{relational database management system} (RDBMS). Desenvolvido inicialmente como um projeto para a CIA (central de inteligência do governo dos EUA), o banco de dados foi evoluindo e liderando o mercado em vendas e inovação \cite{OracleCo5:online}. Seu reino foi soberano até a vinda do fenômeno do Big Data, perdendo espaço para os bancos não relacionais e destruídos \cite{NoSQLWi86:online}.

Interessante que o projeto inicial do SGBDR foi uma ideia de um funcionário da IBM, a qual foi rejeitada \cite{DonodaOr98:online}:
\begin{quotation}
A ideia dos sócios era criar um sistema de gerenciamento de bancos de dados baseado em um estudo feito por um funcionário da IBM que acabou rejeitado, pois a gigante da informática não via nenhum potencial comercial no projeto. Ellison viu, e convenceu a CIA a apostar na ideia, conseguindo um contrato de dois anos para desenvolver a solução, que ele chamou de Oracle (nome com o qual ele viria a rebatizar sua própria companhia).
\end{quotation}



\subsection{Serviço alvo: \C}

O serviço escolhido como alvo do presente plano de marketing é o chamado \C  (\emph{cloud computing} - computação na nuvem). Serviço de \C é a forma de disponibilizar recursos computacionais sob demanda, não apenas sites de internet, mas banco e processamento de dados, sistemas de autenticação e monitoramento etc. \cite{Computa60:online}. São alguns dos principais concorrentes da Oracle, a Amazon Web Services (AWS) \cite{AmazonWe34:online} e a Microsoft Azure \cite{Microsoft12:online}. 

De acordo com o site oficial do serviço de \C da Oracle, dentre diversas alegações de superioridade de seus serviços em relação às concorrentes, duas são as que mais se destacam: serviço \C de segunda geração e por adotar padrões interoperáveis \cite{CloudInf85:online}. A primeira geração de serviço de \C (como a AWS e a Azure) oferece estrutura sob demanda (\emph{on-demand}), \emph{self-service} (liberdade no uso e na combinação entre serviços) e escalabilidade (caso haja necessidade de mais recursos, basta indicar o quanto que a disponibilidade desse é imediata). Agora, quanto a segunda geração, segundo definição da própria Oracle, oferece garantia de performance, previsibilidade de custos e maior controle dos recursos \cite{1WhytheW34:online}. 

\subsection{IaaS, PaaS e SaaS}

São os tipos deserviços oferecidos dentro do serviço de \C conhecidos por IaaS, PaaS e SaaS. O termo \emph{aaS} vem de "\emph{as a service}", ou "como um serviço". Isso significa que um determiando produto é oferecido e consumido como um serviço e não como um produto adquirido por tempo indeterminado. De acordo com \citeonline{Saibaoqu72:online}, líder técnico de \C da IBM, Infraestrutura como Serviço (\emph{Infrastructure as a Service}) é como uma 
\begin{quotation}
(...) camada do sistema operacional e o provedor garante a disponibilidade e a confiabilidade da infraestrutura fornecida
\end{quotation}
e as
\begin{quotation}
(...) empresas que não são proprietárias de um data center veem o IaaS como uma infraestrutura rápida e barata para suas iniciativas de negócios que pode ser expandida ou encerrada, conforme necessário.
\end{quotation}
Quanto à Plataforma como serviço (\emph{IPlatform as a Service}), ainda de acordo com Barabas:
\begin{quotation}
Os provedores de PaaS publicam diversos serviços que podem ser consumidos dentro dos aplicativos. Esses serviços estarão sempre disponíveis e atualizados. O PaaS fornece uma forma muito simples de testar e fazer o protótipo de novos aplicativos. Ele pode economizar ao desenvolver novos serviços e aplicativos. Os aplicativos podem ser liberados mais rapidamente do que o usual para obter feedback do usuário.
\end{quotation}
Já, segundo Barabas, os serviços do tipo Software como serviço (\emph{Software as a Service}) 
\begin{quotation}
(...) são geralmente aceitos por diversas empresas que desejam se beneficiar do uso do aplicativo, sem a necessidade de manter e atualizar a infraestrutura e os componentes. Os aplicativos de e-mail, ERP, colaboração e escritório são as soluções SaaS mais aceitas. A flexibilidade e elasticidade do modelo de SaaS são grandes benefícios.
\end{quotation}

Em suma, software como serviço (SaaS) seria, por exemplo, a disponibilização de um portal de ensino EAD, como o Moodle, edX etc., softwares de gestão, como o Jira, ERPs etc., e outros. Quanto a plataforma como serviço (PaaS), é a disponibilização de um ecosistema de módulos compatíveis com os quais pode-se desenvolver soluções inteiras, não é o fornecimento de soluções prontas como no SaaS, mas o usuário pode unir os módulos para criar uma, sem se preocupar com a infraestrutura (recursos de hardware, rede, internet etc.). No caso de infraestrutura como serviço (IaaS), é disponibilizado os recursos de hardware, tais como computadores virtualizados (ou dedicados), redes privadas e os acessos a rede global (internet). Nisso, é muito comum as empresas adotarem soluções que envolvem diferentes tipos de serviços, como, por exemplo, um serviço peronalizado (como um software propriet
ário) em um computador virtualizado (IaaS) fazendo consultas em um banco de dados como serviço (SaaS), tendo como processos genéricos secundários sendo estanciados sob demanda (PaaS). Pode obter mais detalhes sobre os tipos de serviços citados em \citeonline{SaaSvsPa6:online}. 

\subsection{Objetivo}
A Oracle presta todos serviços, a saber: SaaS, PaaS e IaaS \cite{OqueeOra85:online}. Seu forte em relação ás concorrentes está no PaaS e SaaS, enquanto está investindo e tentanto ser uma das principais prestadores de serviços IaaS. O foco do presente planejamento será em buscar maior aderência a todos os serviços da \C da Oracle apelando para os pontos fortes, a saber: SaaS e PaaS.

\section{Levantamento de dados}

A Oracle é uma empresa sólida e muito bem estabelecida no mercado e isso lhe confere uma grande vantagem: credibilidade e confiânça. Com aquisições de outras empresas de tecnologia (como a Sum Microsystems, dona do Java MySQL e o SO Solaris), cujas propriedade intelectuais impactaram e seguem como altamente relevantes no mercado, desfruta de uma base sólida e confiável de soluções \cite{AlookatO22:online}. O porém está na entrada tardia no mercado do serviço de \C, que se deu em 2015, enquanto a Amazon o fez em 2008 e a Microsoft em 2010, chegando num mercado que, na 
época, já estava saturado de concorrentes. Todavia, com o maior fomento das tecnologias de dados, do uso crescente das técnicas de ciência de dados e de seus derivados, como inteligência artificial e aprendizado de máquina (\emph{machine learning}), a demanda por soluções escaláveis e com custo reduzido reaqueceu o mercado do serviço de \C. A geração cada vez maior de dados e a necessidade de velocidade de processamento e acesso, bem como agilidade no desenvolvimento, manutenção e melhora das soluções, levantou a demanda pelos produtos como serviço (os \emph{aaS}), principalmente os SaaS e PaaS (software e plataforma como serviço).

Esses pontos são de interesse do presente plano: tonar as soluções de \C da Oracle, principalmente os SaaS e PaaS, mais conhecidos entre os profissionais da área de tecnologia da informação, tendo como principais alvos os cientistas de dados, engenheiros de dados e engenheiros de software. Com efeito, esses três tipos de profissionais são os que desenham, constroem e dão manutenção nas soluções digitais de qualquer empresa. Por se tratar de um produto B2B, faz-se necessário convencer os tomadores de decisão dentro da empresa e esse profissionais são os maiores influenciadores. Gestores, em geral, estão mais preocupados com a sustentabilidade do negócio e, de forma mais estratégica, com a eficiência operacional, para tanto, decisões de detalhes referentes às tecnologias utilizadas caberá aos técnicos. Com efeito, sendo o serviço de Oracle eficiente, estável e de alta credibilidade, sobre a necessidade de convencer os técnicos de que vale a pena migrarem ou desenvolverem novas soluções na \C da Oracle.

Já é bem estabelecido que eventos de tecnologia são altamente benéficos para os profissionais \cite{10Reason10:online} e para as empresas (não apenas de tecnologia, mas que possuem desenvolvimento interno de tecnologia) \cite{5keybene63:online}. Com efeito, a principal ação proposta é a de promover eventos tecnológicos e participar nos já existentes e consolidados, promovendo cursos práticos, seminários e oficinas, todos baseados nos produtos da \C da Oracle. Deseja-se com isso influenciar os técnicos e gerentes para que passem a colocar os serviços Oracle como uma alternativa mais eficiente, apelando para os benefícios da segunda geração dos serviços de \C.

Além dos principais eventos específicos de TI, como as conferências anuais (uma lista para 2020 pode ser encontrada em \citeonline{2020Prin23:online}), deve-se considerar os eventos de desenvolvedores (engenheiros de cientistas em geral), como o \emph{The Developer's Conference} (TDC) promovido pela \citen{TheDevCo57:online}, o \emph{PyData} da \citen{PyData96:online} e os diferentes eventos do \emph{Meetup} sobre os assuntos correlatos \cite{Meetup:online}. Objetiva-se atingir o mesmo efeito que a Amazon conseguiu sendo um dos pioneiros que é de gerar um \emph{buzz} marketing eficiente, a qual, hoje, possui milhares de tutoriais e cursos online, apresentação de casos de uso em eventos de tecnologia (fomentado por terceiros) etc.


\section{Implementação}

Para que as ações propostas nos eventos de tecnologias sejam bem sucedidos é necessário que essas sejam elaboradas, portanto o primeiro passo a ser dado é na obtenção e formação de profissionais que serão os autores dos recursos de divulgação como os cursos práticos, seminários e oficinas. Profissionais internos e contratados para essa finalidade podem receber uma formação adicional para serem autores dos recursos de divulgação, usando sua experiência prévia. Pode-se contratar por evento (temporário) profissionais de clientes da Oracle, os quais já usam os seriços de \C. 

Por meio de programas gratuitos de ensino imersivo (\emph{bootcamp}), como os da \citeonline{IGTI:online}, pode-se atrair mais interessados, pois cursos desse tipo oferecem certificados que não são apenas de aprticipação, mas conferem graus de acordo com desepenho, entrega de projetos etc. Além de aumentar a base de profissionais com conhecimento e experiência prática nos serviços de \C da Oracle, fomentando o \emph{buzz} marketing, pode servir para fins de recrutamento, aumentando a base de autores para divulgação nos eventos de tecnologia.

Os recursos criados pelos autores podem servir para mais de um evento, por exemplo, os eventos de ciência de dados e aprendizado de máquina do \emph{Meetup} e do \emph{PyData} são perfeitamente compatíveis. O passo seguinte, portanto, é definir para quais eventos cada recurso tem maiores chances de sucesso, ou seja, maior conversão dos técnicos. Investigar o público alvo de vada evento e verificar se os recursos se adequam.

\section{Análise e controle}

Para cada evento, coleta-se os dados dos participantes, dentre os dados mais grais, como nome e idade, pode-se registrar onde trabalha, a posição, se possui cargo de gerência ou não, se tem poder de decisão etc. Após cada evento, um questionário é enviado para obter o nível de satisfação com o evento em si, com o instrutor e com as tecnologias e serviços experimentados. Esse formulário, além de indicar a qualidade do recurso usado, pode dar indícios sobre como são as impressões a respeito do serviço de \C da Oracle. Esse pontos podem ser avaliados e ajustado nos próximos recursos a fim de atenuar (ou eliminar) os efeitos negativos e acentuar (ou manter) os positivos.

Após um determinado tempo, de uma ou duas semanas, a segunda ação após os eventos é entrar em contato com os participantes propondo ações de divulgação dos serviços de \C dentro na empresa do participante (ou com a participação de mais colegas em teleconferência, pois podem estar em teletrabalho). Caso seja necessário, até a repetição de um dos recursos de divulgação, como cursos ou seminários. Nesse caso, há maiores chances de algum dos participantes possuir maior influência em decisões internas.

A fim de se monitorar a longo prazo, é prevista um contato recorrente com os participantes de todos os recrusos de divulgação, em cada 2 ou 3 meses, para mensurar a taxa de conversão (quantos dos participantes passaram a usar os erviços de \C da Oracle). Além disso, obtém-se possíveis informações sobre a razões que não levaram a adoção dos serviços.

\section{Resumo esquemático}

\subsection{SWOT}

De acordo com \citeonline{MarketingTeixeira}, SWOT é
\begin{quotation}
(...) uma sigla formada pelas palavras inglesas \emph{strengths} (forças), \emph{weakness} (fraquezas), \emph{opportunities} (oportunidades) e \emph{threats} (ameaças). A Análise Swot propõe um exame completo desses
elementos, para que se possa fazer um diagnóstico preciso da empresa.
\end{quotation}

\subsubsection{Strengths}
\begin{itemize}
\item Maior qualidade e segurança, por ser um serviço de nuvem computacional de segunda geração;
\item o preço competitivo e com tarifação diferenciada em relação aos concorrentes;
\item ao tonar os orçamento mais previsível;
\item e a credibilidade pelo tempo de mercado, pela quantidade de clientes e pelo portfólio de soluções bem estabelecidas no mercado.
\end{itemize}

\subsubsection{Weakness}

\begin{itemize}
\item A interface de usuário não é intuitiva;
\item ainda há pouca documentação oficial sobre os serviços como descrição das funcionalidade e exemplos práticos. 
\end{itemize}

\subsubsection{Opportunities}

\begin{itemize}
\item Demanda crescente por serviços escaláveis rapidamente, com custos de arrancada baixos, com rápida disponibilidade, causada por uma demanda maior para o tratamento e monetização de dados. 
\item Projetos em fase inicial, geralmente, são facilmente viabilizados e podem entrar em produção mais rapidamente com serviços de \C. 
\item Crescente popularização dos serviços de \C entre as as empresas e os profissionais técnicos, com a promessa de redução de custos, facilidade de manutenção, monitoramento, contingenciamento, bem como mais agilidade no desenvolvimento e evolução dos produtos.
\end{itemize}
\subsubsection{Threats}

\begin{itemize}
\item Possíveis incompatibilidades entre os produtos dos \emph{prospects}, reduzindo a atratividade.
\item A diferença de preço ainda pode não compensar para os \emph{prospects} que já possuem sistemas legados.
\end{itemize}

\subsection{Plano de ação}

Fomentar o \emph{buzz} marketing do serviço de \C da Oracle por meio de cursos e seminários. Fazer contatos recorrentes com os participantes para prospecção de novos clientes. Em suma, criar e mante ruma comunidade de profissionais com conhecimento e experiência nos serviços de \C da Oracle.

\subsection{Plano de manutenção}

Investigar junto aos participantes os pontos fortes e fracos do serviço e encaminhar as melhorias necessárias. Estabelecer um ciclo de participações em eventos e formação continuada.

\section{Considerações finais}

Com o presente plano pretende-se atacar o principal ponto fraco do serviço de \C da Oraclo: a pouca quantidade de profissionais com experiência e interesse no serviço. É sabido que os profissionais técnicos são os principais influenciadores na tomada de decisões sobre quando e qual serviço de \C a empresa deve adotar, e esse alinhamento entre as áreas de gestão, financeiro e técnica é cada vez mais reforçado, ou seja, se a área técnica reprova um determinado prestador de serviço, as chances desse são mínimas. Observado o histórico de diversas tecnologias, produtos e serviços, vemos o quão importante o \emph{buzz} marketing pode ser importante.

Um exemplo clássico DE \emph{buzz} marketing é o sistema operacional Windows da Microsoft, o qual teve uma alta demanda pois a sua aquisição para aprendizado doméstico era fácil devido a facilidade em piratear o uso. Com uma quantidade tão grande de indivíduos com experiência prévia no Windows, é comum que esse SO e os demais produtos que são compatíveis com ele (como o MS Office) sejam a escolha dentro das empresas, nas quais, a pirataria é bem mais restrita.


\addto{\captionsbrazil}{% caso use \usepackage[brazil]{babel}
\renewcommand{\bibname}{Refer\^{e}ncias Teses}
}
\bibliographystyle{renote}
\bibliography{References/references}

\end{document}
\section{Notas}
profa dicas: